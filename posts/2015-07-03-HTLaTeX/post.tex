\documentclass{article}
\usepackage{hyperref}
\usepackage{url}
\usepackage{natbib}

\title{Using {\LaTeX} with Hakyll: 2}
\author{Takayuki Muranushi}

\begin{document}

\maketitle

This HTML is generated from {\LaTeX} source via {\tt htlatex}
and then embedded into Hakyll-hosted blog. The use of LaTeX will allow us the use of
complex expressions e.g. $G_{\mu,\nu}=\frac{8\pi G}{c^4}T_{\mu,\nu}$ and like Equation (\ref{eq:1}).

\begin{eqnarray}
\frac{\partial{\vec v}}{\partial t} &=& \vec \nabla s \label{eq:1}
\end{eqnarray}



This attempt seems to basically working, and the few
glitches of
% note: I think it's okay to use the absolute URL, since this url also goes into the pdf file.
\href{http://nushio3.github.io/posts/2014-10-09-LaTeX2HTML/post/post.html}{the previous attempt},
like the duplicated title is no more there. Fixing these was easy because
all I had to do is to add some Markdown metadata directly to HTML.

The sad thing is the poor quality of the rendering of the math equations. Some day I'll try out the
\href{http://tex.stackexchange.com/questions/44486/pixel-perfect-vertical-alignment-of-image-rendered-tex-snippets}{technique by Todd}.


I am further trying to use this technology in combination with Haskell
DSLs such as
\href{http://hackage.haskell.org/package/authoring}{authoring},
units \citep{muranushi2014experience} ,
units-of-measure plugin \citep{gundry2015typechecker}
in order to write
physical discussions in Haskell and LaTeX.

\bibliographystyle{abbrvnat}
%\bibliographystyle{siam}
%\bibliographystyle{plainnat}
%\bibliographystyle{plain}
\bibliography{references}

The content of this page is also availabe as a pdf document:
\href{http://nushio3.github.io/posts/2015-07-03-HTLaTeX/dist/post.pdf}{Using {\LaTeX} with Hakyll: 2}.

\end{document}
