\documentclass{article}
\usepackage{hyperref}
\usepackage{url}
\usepackage{natbib}

\title{Using {\LaTeX} with Hakyll: 2}
\author{Takayuki Muranushi}

\begin{document}

\maketitle

When I wish to write academic articles on my blog, I need to get many things done; the equations, the figures,
the hyperlinks and the citations. Since LaTeX, after 35 years, is still an excellent tool for authoring with all
those features, I decided it's good to have the option to write my blog in LaTeX. Here I'll leave the note how to write
Hakyll articles using LaTeX, for people who want to do so, including future me. The source codes are published at
\href{https://github.com/nushio3/nushio3.github.io}{the github repository}.


This HTML is generated from {\LaTeX} source via {\tt htlatex}
and then embedded into Hakyll-hosted blog.
The article is generated by two steps. First, you {\tt make} the article at
\href{https://github.com/nushio3/nushio3.github.io/tree/source/posts/2015-07-03-HTLaTeX}{the post-specific directory},
which will invoke  {\tt htlatex} and prepare the HTML for the post. Next,  {\tt site.hs} at
\href{https://github.com/nushio3/nushio3.github.io}{the top directory} will take the HTML, apply the blog template (headers, menus and so on) and integrate it into the blog. The latter process is usually automated as I run {\tt ./site.hs watch}.
When everything is done, I \href{https://github.com/nushio3/nushio3.github.io/blob/source/deploy.sh}{deploy} the entire blog to
\href{http://nushio3.github.io/}{github.io} .


The use of LaTeX will allow us the use of
complex expressions e.g. $G_{\mu,\nu}=\frac{8\pi G}{c^4}T_{\mu,\nu}$ and like Equation (\ref{eq:1}).

\begin{eqnarray}
\frac{\partial{\vec v}}{\partial t} &=& \vec \nabla s \label{eq:1}
\end{eqnarray}



This attempt seems to be basically working, and the few
glitches of
% note: I think it's okay to use the absolute URL, since this url also goes into the pdf file.
\href{http://nushio3.github.io/posts/2014-10-09-LaTeX2HTML/post/post.html}{the previous attempt},
like the duplicated title is no more there. Fixing these were easy because
all I had to do is to have
\href{https://github.com/nushio3/nushio3.github.io/blob/source/posts/2015-07-03-HTLaTeX/metadata.md}{some Hakyll metadata} and
\href{https://github.com/nushio3/nushio3.github.io/blob/source/posts/2015-07-03-HTLaTeX/Makefile}{append them directly to HTML}.

The sad thing is the still poor quality of the the math equations rendering,
although a little improvement is seen from the LaTeX2HTML version.
Some day I'll try out the
\href{http://tex.stackexchange.com/questions/44486/pixel-perfect-vertical-alignment-of-image-rendered-tex-snippets}{technique by Todd}.


I am further trying to use this technology in combination with Haskell
DSLs such as
\href{http://hackage.haskell.org/package/authoring}{authoring},
units \citep{muranushi2014experience} ,
units-of-measure plugin \citep{gundry2015typechecker}
in order to write
physical discussions in Haskell and LaTeX.

\bibliographystyle{abbrvnat}
%\bibliographystyle{siam}
%\bibliographystyle{plainnat}
%\bibliographystyle{plain}
\bibliography{references}

The content of this page is also availabe as a pdf document:
\href{http://nushio3.github.io/posts/2015-07-03-HTLaTeX/dist/post.pdf}{Using {\LaTeX} with Hakyll: 2}.

\end{document}
